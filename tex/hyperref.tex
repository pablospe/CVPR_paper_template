                        %%%%%   HYPERREF  %%%%%

% % If you comment hyperref and then uncomment it, you should delete
% % egpaper.aux before re-running latex.  (Or just hit 'q' on the first latex
% % run, let it finish, and you should be clear).
% \usepackage[breaklinks=true,bookmarks=false]{hyperref}

\usepackage[pagebackref=true]{hyperref}


\definecolor{linkcolour}{rgb}{0,0.2,0.6}
\definecolor{citecolour}{rgb}{0,0.6,0.2}
\definecolor{urlcolour} {rgb}{0.8,0,0.8}

\hypersetup{
%     bookmarks=false,
    bookmarksopen=false,
    bookmarksnumbered=true,
    pdfpagemode=UseOutlines,  % None, UseThumbs, UseOutlines (show bookmarks), FullScreen
    colorlinks=true,          % color link text, not a box around them.
    linkcolor=linkcolour,     % color for normal internal links.
    anchorcolor=black,        % color for anchor text.
    citecolor=citecolour,     % color for bibligraphical citations in text.
    filecolor=urlcolour,      % color for URLs which open local files.
    menucolor=blue,           % color for Acrobat menu items.
    urlcolor=urlcolour,       % color for linked URLs.
    breaklinks=true,          % Allows link text to break across.
    pdfstartview={FitBH},
    pdfview={FitBH},
    % pdftitle={\thetitle},
    % pdfauthor={\theauthor},
}

% \definecolor{linkcolour}{rgb}{0,0.2,0.6}
\definecolor{citecolour}{rgb}{0,0.6,0.2}
\definecolor{urlcolour} {rgb}{0.8,0,0.8}
\hypersetup{
    linkcolor=linkcolour,
    citecolor=citecolour,
    filecolor=urlcolour,
    urlcolor=urlcolour
}


% https://tex.stackexchange.com/questions/54946/how-to-break-long-url-in-an-item#254734
\def\UrlBreaks{\do.\do\/\do-}
